\documentclass[]{article}
\usepackage{lmodern}
\usepackage{amssymb,amsmath}
\usepackage{ifxetex,ifluatex}
\usepackage{fixltx2e} % provides \textsubscript
\ifnum 0\ifxetex 1\fi\ifluatex 1\fi=0 % if pdftex
  \usepackage[T1]{fontenc}
  \usepackage[utf8]{inputenc}
\else % if luatex or xelatex
  \ifxetex
    \usepackage{mathspec}
  \else
    \usepackage{fontspec}
  \fi
  \defaultfontfeatures{Ligatures=TeX,Scale=MatchLowercase}
\fi
% use upquote if available, for straight quotes in verbatim environments
\IfFileExists{upquote.sty}{\usepackage{upquote}}{}
% use microtype if available
\IfFileExists{microtype.sty}{%
\usepackage{microtype}
\UseMicrotypeSet[protrusion]{basicmath} % disable protrusion for tt fonts
}{}
\usepackage[vcentering,landscape,a4paper, top=1.54cm,bottom=1.54cm,
left=2.54cm,right=2.54cm]{geometry}
\usepackage{hyperref}
\hypersetup{unicode=true,
            pdfborder={0 0 0},
            breaklinks=true}
\urlstyle{same}  % don't use monospace font for urls
\usepackage{graphicx,grffile}
\makeatletter
\def\maxwidth{\ifdim\Gin@nat@width>\linewidth\linewidth\else\Gin@nat@width\fi}
\def\maxheight{\ifdim\Gin@nat@height>\textheight\textheight\else\Gin@nat@height\fi}
\makeatother
% Scale images if necessary, so that they will not overflow the page
% margins by default, and it is still possible to overwrite the defaults
% using explicit options in \includegraphics[width, height, ...]{}
\setkeys{Gin}{width=\maxwidth,height=\maxheight,keepaspectratio}
\IfFileExists{parskip.sty}{%
\usepackage{parskip}
}{% else
\setlength{\parindent}{0pt}
\setlength{\parskip}{6pt plus 2pt minus 1pt}
}
\setlength{\emergencystretch}{3em}  % prevent overfull lines
\providecommand{\tightlist}{%
  \setlength{\itemsep}{0pt}\setlength{\parskip}{0pt}}
\setcounter{secnumdepth}{0}
% Redefines (sub)paragraphs to behave more like sections
\ifx\paragraph\undefined\else
\let\oldparagraph\paragraph
\renewcommand{\paragraph}[1]{\oldparagraph{#1}\mbox{}}
\fi
\ifx\subparagraph\undefined\else
\let\oldsubparagraph\subparagraph
\renewcommand{\subparagraph}[1]{\oldsubparagraph{#1}\mbox{}}
\fi

%%% Use protect on footnotes to avoid problems with footnotes in titles
\let\rmarkdownfootnote\footnote%
\def\footnote{\protect\rmarkdownfootnote}

%%% Change title format to be more compact
\usepackage{titling}

% Create subtitle command for use in maketitle
\providecommand{\subtitle}[1]{
  \posttitle{
    \begin{center}\large#1\end{center}
    }
}

\setlength{\droptitle}{-2em}

  \title{}
    \pretitle{\vspace{\droptitle}}
  \posttitle{}
    \author{}
    \preauthor{}\postauthor{}
    \date{}
    \predate{}\postdate{}
  
\usepackage[brazil]{babel} \usepackage[utf8]{inputenc} \usepackage{csvtools} \usepackage{fancyhdr} \usepackage[usenames,svgnames,dvipsnames,table]{xcolor} \usepackage[none]{hyphenat} \usepackage{setspace}

\begin{document}

\thispagestyle{empty} 
\begin{minipage}{\textwidth}
            \includegraphics[width=\textwidth]{imagens/header.jpg}

            \sffamily
            \bigskip
            \bigskip
\end{minipage}

\begin{center}
                {\Huge \textbf{CERTIFICADO}}\\
                \bigskip
                \bigskip
            \end{center}

\begin{center}
                \begin{minipage}{0.8\textwidth}
                    {\Large Certificamos que \emph{Alexandre }\ esteve presente no 
                    \texttt{Lançamento do livro "Dom Casmurro"},  realizado em \texttt{05 de dezembro de 1899} na cidade de 
                    \texttt{Paris}, na qualidade de \texttt{visitante}.}
                \end{minipage}
            \end{center}

\begin{verbatim}
\begin{center}
    \vspace{1.2cm}
    \begin{minipage}
\end{verbatim}

\{\textwidth\} \center
\includegraphics[width=\textwidth]{imagens/assinatura.png}
\textbackslash end\{minipage\}

\begin{verbatim}
            \Large {\rule{7.0cm}{0.5pt}\\
            \texttt{Machado de Assis}\\
            Autor de Dom Casmurro\\}
            \vspace{0.5cm}
        \end{center}    
        
\end{verbatim}

1.30082018.1

\begin{minipage}
{\textwidth}
            \includegraphics[width=\textwidth]{imagens/footer.jpg}

\end{minipage}


\end{document}
